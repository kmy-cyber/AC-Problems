\documentclass{article}
\usepackage{listings}

\begin{document}

\title{Pseudocodigo problema 75}
\author{Equipo 11}
\date{\today}

\maketitle

\section{Introducción}
En este articulo mostraremos el pseudocodigo de como resolver el problema de validacion de fecha.


\section{Código de programación}


\begin{lstlisting}[language=Python]
def hello_world():
	print("Hola, mundo!")

def verificar_fecha(d, m, a):
	if m < 1 or m > 12:
		return False
	if d < 1 or d > 31:
		return False
	if m == 2:
		if a % 4 == 0 and (a % 100 != 0 or a % 400 == 0):
			if d > 29:
				return False
		elif d > 28:
			return False
	elif m in [4, 6, 9, 11]:
		if d > 30:
			return False
	return True

d = 31
m = 12
a = 2022

if verificar_fecha(d, m, a):
	print("La fecha es valida")
else:
	print("La fecha no es valida")

hello_world()
\end{lstlisting}

\section{Conclusiones}

Un ejemplo de pseudocódigo en Python de como validar una fecha. Trataremos de hacer algo similar en una maquina de estados algoritmica en logisim.

\end{document}
\documentclass{article}
\usepackage{listings}
\usepackage{xcolor}
\usepackage{graphicx}

\definecolor{codegreen}{rgb}{0,0.6,0}
\definecolor{codegray}{rgb}{0.5,0.5,0.5}
\definecolor{codepurple}{rgb}{0.58,0,0.82}
\definecolor{backcolour}{rgb}{0.95,0.95,0.92}

\lstdefinestyle{mystyle}{
	backgroundcolor=\color{backcolour},
	commentstyle=\color{codegreen},
	keywordstyle=\color{magenta},
	numberstyle=\tiny\color{codegray},
	stringstyle=\color{codepurple},
	basicstyle=\footnotesize,
	breakatwhitespace=false,
	breaklines=true,
	captionpos=b,
	keepspaces=true,
	numbers=left,
	numbersep=5pt,
	showspaces=false,
	showstringspaces=false,
	showtabs=false,
	tabsize=2
}

\lstset{style=mystyle}

\title{Menor sufijo a agregar para convertir en Palindromo (Problema 64)}



\begin{document}
	
	\maketitle
	
	\section{Descripcion del Ejercicio}
	

Un palíndromo es una palabra, frase, número o secuencia que se lee igual hacia adelante que hacia atrás, sin importar la dirección en la que se lea. Los palíndromos pueden ser de una sola palabra, como "11011" o "12321".

Dado un patrón en forma de bits. Devolver el menor sufijo que habría que agregarle a dicho patrón para que fuera palíndromo.

\textbf{Salida}

Para la salida debe imprimir el menor sufijo, separando cada dígito por un espacio.

Por ejemplo, para $L = [1, 0, 1, 1]$ debería imprimir:

> 0 1

\textbf{Logisim}

Se dispondrá en \textit{INPUT} los datos de entrada a partir de la dirección $0$. La entrada se estructura de la siguiente forma:

- $w_0$: $n$ (tamaño de la lista $L$)
- $w_{1:n}$ : $L$

\textbf{SASM}

En la sección \texttt{.data} se deben definir los valores de entrada de la siguiente forma:

- \texttt{n} : un número de tamaño \texttt{dd} que representa al tamaño de la lista $L$
- \texttt{array} : un array de números de tamaño \texttt{dd} que representa $L$

Por ejemplo, un posible encabezado podría ser:

\begin{verbatim}
	section .data
	n dd 4
	array dd 1, 0, 1, 1
\end{verbatim}
	
	\section{Pseudocódigo}
	
	\begin{lstlisting}[language=Python, caption=Función  en Python]
	# Funcion para saber si es palindromo
	def esPalindromo(arr, inicio, n):
		fin = n - 1
		while inicio < fin:
			if arr[inicio] != arr[fin]:
				return False
			inicio = inicio + 1
			fin = fin - 1
	return True
	
	# Funcion para imprimir el sufijo minimo que necesita ser agregado al inicio 
	# para hacer el arreglo un palindromo.
	def imprimirSufijoParaPalindromo(arr, n):
	
	# Encuentra la posicion inicial desde donde el sufijo debe
	# ser un palindromo
		while i < n:
			if esPalindromo(arr, i, n):
				break
			i = i+1
	
	# Imprimir el sufijo necesario en orden inverso, que debe ser agregado al inicio
	i = i-1
	while i>=0:
		print(arr[i], end =" ")
	
	arr = [1, 0, 1, 1]
	n = len(arr)
	
	print("El sufijo necesario para convertir el arreglo en un palindromo es:", end=" ")
	imprimirSufijoParaPalindromo(arr, n)
	print()
		
	\end{lstlisting}
	
	\section{Código en Ensamblador}
	
	\begin{lstlisting}[language={[x86masm]Assembler}, caption=Código Tribonacci en Ensamblador]
	%include "io.inc"
	
	section .data
	arr dd 1, 0, 1, 1, 0, 0    ; Definir el arreglo.
	n dd 6                  ; Longitud del arreglo.
	
	section .bss
	; Reserva de espacio para variables si es necesario.
	
	section .text
	global CMAIN
	
	CMAIN:
	mov ebp, esp; for correct debugging
	mov edi, arr        ; Primer argumento de 'imprimirSufijoParaPalindromo', la direccion de arr.
	mov esi, [n]        ; Segundo argumento de 'imprimirSufijoParaPalindromo', el tamaño de arr.
	call _imprimirSufijoParaPalindromo
	
	xor eax, eax
	ret
	
	_esPalindromo:
	mov ecx, esi                  ; copiamos n a ecx.
	dec ecx                       ; fin = n - 1.
	mov ebx, eax                  ; copiamos inicio a ebx (inicio de comparación).
	
	palindromo_loop:
	cmp ebx, ecx                  ; Comparamos inicio con fin.
	jge .esPalindromoRetornoVerdadero ; Si inicio >= fin, es palindromo.
	mov eax, [edi + ebx*4]        ; Valor en inicio.
	cmp eax, [edi + ecx*4]        ; Valor en fin.
	jne .esPalindromoRetornoFalso
	inc ebx                       ; Incrementamos inicio.
	dec ecx                       ; Decrementamos fin.
	jmp palindromo_loop
	.esPalindromoRetornoFalso:
	mov eax, 0                     ; Retornar falso (0).
	ret
	.esPalindromoRetornoVerdadero:
	mov eax, 1                     ; Retornar verdadero (1).
	ret
	
	_imprimirSufijoParaPalindromo:
	xor ecx, ecx                   ; i = 0.
	
	find_palindromo_prefix_loop:
	cmp ecx, esi                   ; Comparamos i con n.
	je .find_palindromo_prefix_done ; si i == n hemos terminado.
	mov edx, esi                   ; Argumento n para esPalindromo.
	mov eax, ecx                   ; Argumento inicio para esPalindromo.
	push ecx
	call _esPalindromo
	pop ecx
	test eax, eax                  ; Verificamos si esPalindromo retorno verdadero.
	jz .increment_i                ; Si no, incrementamos i y continuamos.
	jmp .find_palindromo_prefix_done ; Si hemos terminado.
	
	.increment_i:
	inc ecx
	jmp find_palindromo_prefix_loop
	
	.find_palindromo_prefix_done:
	dec ecx ; Ajustamos para el indice correcto del ultimo elemento no palindromo.
	print_prefix:
	cmp ecx, -1 ; Verificamos si hemos terminado con todos los elementos.
	jle .done_printing
	
	PRINT_DEC 4, [arr + ecx*4]
	
	dec ecx ; Mover al siguiente elemento para imprimir.
	jmp print_prefix
	
	.done_printing:
	ret
	\end{lstlisting}
	
	
	
\end{document}
